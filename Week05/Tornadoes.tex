% Options for packages loaded elsewhere
\PassOptionsToPackage{unicode}{hyperref}
\PassOptionsToPackage{hyphens}{url}
%
\documentclass[
]{article}
\usepackage{amsmath,amssymb}
\usepackage{iftex}
\ifPDFTeX
  \usepackage[T1]{fontenc}
  \usepackage[utf8]{inputenc}
  \usepackage{textcomp} % provide euro and other symbols
\else % if luatex or xetex
  \usepackage{unicode-math} % this also loads fontspec
  \defaultfontfeatures{Scale=MatchLowercase}
  \defaultfontfeatures[\rmfamily]{Ligatures=TeX,Scale=1}
\fi
\usepackage{lmodern}
\ifPDFTeX\else
  % xetex/luatex font selection
\fi
% Use upquote if available, for straight quotes in verbatim environments
\IfFileExists{upquote.sty}{\usepackage{upquote}}{}
\IfFileExists{microtype.sty}{% use microtype if available
  \usepackage[]{microtype}
  \UseMicrotypeSet[protrusion]{basicmath} % disable protrusion for tt fonts
}{}
\makeatletter
\@ifundefined{KOMAClassName}{% if non-KOMA class
  \IfFileExists{parskip.sty}{%
    \usepackage{parskip}
  }{% else
    \setlength{\parindent}{0pt}
    \setlength{\parskip}{6pt plus 2pt minus 1pt}}
}{% if KOMA class
  \KOMAoptions{parskip=half}}
\makeatother
\usepackage{xcolor}
\usepackage[margin=1in]{geometry}
\usepackage{graphicx}
\makeatletter
\def\maxwidth{\ifdim\Gin@nat@width>\linewidth\linewidth\else\Gin@nat@width\fi}
\def\maxheight{\ifdim\Gin@nat@height>\textheight\textheight\else\Gin@nat@height\fi}
\makeatother
% Scale images if necessary, so that they will not overflow the page
% margins by default, and it is still possible to overwrite the defaults
% using explicit options in \includegraphics[width, height, ...]{}
\setkeys{Gin}{width=\maxwidth,height=\maxheight,keepaspectratio}
% Set default figure placement to htbp
\makeatletter
\def\fps@figure{htbp}
\makeatother
\setlength{\emergencystretch}{3em} % prevent overfull lines
\providecommand{\tightlist}{%
  \setlength{\itemsep}{0pt}\setlength{\parskip}{0pt}}
\setcounter{secnumdepth}{-\maxdimen} % remove section numbering
\ifLuaTeX
  \usepackage{selnolig}  % disable illegal ligatures
\fi
\usepackage{bookmark}
\IfFileExists{xurl.sty}{\usepackage{xurl}}{} % add URL line breaks if available
\urlstyle{same}
\hypersetup{
  pdftitle={P1 - An Analysis on the Effects of Tornadoes in 2022},
  pdfauthor={Arun Mahadevan Sathia Narayanan \textbar{} as235872 \textbar{} arunmsnarayanan@utexas.edu},
  hidelinks,
  pdfcreator={LaTeX via pandoc}}

\title{P1 - An Analysis on the Effects of Tornadoes in 2022}
\author{Arun Mahadevan Sathia Narayanan \textbar{} as235872 \textbar{}
\href{mailto:arunmsnarayanan@utexas.edu}{\nolinkurl{arunmsnarayanan@utexas.edu}}}
\date{Date Published: 2024-09-24}

\begin{document}
\maketitle

{
\setcounter{tocdepth}{2}
\tableofcontents
}
\begin{figure}
\centering
\includegraphics{https://www.vercounty.org/wp-content/uploads/2020/09/tornado-banner.jpg}
\caption{Figure I.1: A tornado caused by a supercell. Source:
(\url{https://www.vercounty.org/news-item/tornado-watches-vs-warnings/})}
\end{figure}

\section{INTRODUCTION}\label{introduction}

Welcome! Hold onto your hat --- this session might just sweep you off
your feet, but don't worry, I'll keep it all under control\ldots{}
mostly!

As the title states, the data set being analyzed in this report is the
tornadoes of 2022. More specifically, we look at the magnitude of the
tornadoes, if there were fatalities/injuries, the estimated property
loss (in US dollars), and the size.

The data was retrieved from the NOAA
(\url{https://www.spc.noaa.gov/wcm/\#data}).\\
The variable of interest (dependent variable) that I will be exploring
in this report is the estimated property loss, with the predictor
variables (independent variables) being the state the tornado was in,
the magnitude of the tornado, and the distance the tornado traveled. We
will be analyzing these for the 591 tornadoes the data was collected
for.

And yes, you have to grit your teeth and push your way through the
ridiculous puns.

\section{OUTCOME VARIABLE}\label{outcome-variable}

Now that we have addressed what exactly to look at, let's visualize the
dataset we have to analyze:

\includegraphics{Tornadoes_files/figure-latex/unnamed-chunk-1-1.pdf}

\subparagraph{Figure O.1: A boxplot visualization of the spread of
property loss
data.}\label{figure-o.1-a-boxplot-visualization-of-the-spread-of-property-loss-data.}

Looking at this boxplot, we can see that many of the tornadoes have done
property damage in the lower half of the data. To better analyze this,
let's see the actual numbers.

\begin{verbatim}
##      Min.   1st Qu.    Median      Mean   3rd Qu.      Max. 
##      1000     37500    100000   2099165    347500 220000000
\end{verbatim}

\subparagraph{Figure O.2: The number-view of the spread of the property
loss
data.}\label{figure-o.2-the-number-view-of-the-spread-of-the-property-loss-data.}

From these numbers we can see the hard data:\\
The values all the way up until the 3rd Quartile (meaning 75\% of the
total data) go only until \$347,500. The average property loss caused by
these 591 tornadoes is \$2,099,165 (and of course, this is mainly due to
the mean being high affected by the outliers in the data). The maximum
damage caused by any tornado in the United States in 2022 was
\$220,000,000.

\begin{itemize}
\tightlist
\item
  Why did the tornado say to the data scientist?\\
  ``Hold on tight, I'm about to cause a \(category \space shift\) in
  your results!''
\end{itemize}

\section{BIVARIATE ANALYSES}\label{bivariate-analyses}

Now, we get to the good stuff.

Since we got an understanding of what exactly we're looking at here,
let's see how exactly property loss is related to the states, the
magnitudes, and the distances traveled.

\subsubsection{State vs.~Property Loss}\label{state-vs.-property-loss}

\includegraphics{Tornadoes_files/figure-latex/unnamed-chunk-3-1.pdf}

\subparagraph{Figure B.1: The relationship between the state and the
property
loss.}\label{figure-b.1-the-relationship-between-the-state-and-the-property-loss.}

Here, we have a boxplot made for the property loss per state. From the
visualization, we can see that a large cluster of the data for each
state is in the lower range of the data. The largest property loss, as
observed from the boxplot distributions, was in Iowa (IA). If you want
to know more about this tornado, you can go here:
(\url{https://en.wikipedia.org/wiki/Tornado_outbreak_of_March_5–7,_2022}).
To take a closer look at the data for the lower ranges, let's set a
bound on the property loss value to \$53,000,000.

\includegraphics{Tornadoes_files/figure-latex/unnamed-chunk-4-1.pdf}

From this graph, we can see the distribution of the data a little
better. Here, we can see that Kansas (KS) had the largest spread (though
Michigan (MI) had the largest property loss, it is the only data point,
and thus has no spread), with a minimum of around \$10,000,000 and a
maximum of around \$42,000,000.

The main reason for performing this state comparison is to analyze
whether or not the Tornado Alley is just a myth or not. The Tornado
Alley consists of Texas (TX), Oklahoma (OK), Kansas (KS), Nebraska (NE),
and South Dakota (SD) {[}the South Dakota was all N/A and was thus
excluded from the boxplot distributions{]}. When we look at these states
specifically (again, excluding SD) we see that from the lower portions
Texas is larger than the other data points, Oklahoma not so much,
Nebraska has a distribution a bit higher than the other states, and
Kansas, as we previously discussed, has the largest data spread. When
comparing these states to the others, the Tornado Alley seems to be a
realistic area for a larger number of tornadoes. If you want to learn
more about the Tornado Alley, go here:
(\url{https://en.wikipedia.org/wiki/Tornado_Alley}).

\subsubsection{Magnitude vs.~Property
Loss}\label{magnitude-vs.-property-loss}

The next bivariate distribution to analyze the relationship between the
magnitude of the tornado and the property loss incurred.

\includegraphics{Tornadoes_files/figure-latex/unnamed-chunk-5-1.pdf}

\subparagraph{Figure B.2: The relationship between the magnitude and the
property
loss.}\label{figure-b.2-the-relationship-between-the-magnitude-and-the-property-loss.}

We can also make this a boxplot, and from the data, we can analyze that
the magnitude which cased the largest property loss was indeed a
magnitude 4 tornado. But we can visualize that the property loss done by
the magnitude 3's are less than the property loss done by the magnitude
2's. To properly analyze the relationship, let's find the correlation:

\begin{verbatim}
## [1] 0.3716175
\end{verbatim}

\paragraph{About Correlation}\label{about-correlation}

How correlation works is that the decimal value has to be between -1 and
1, with -1 being a perfectly negative correlation and 1 being a
perfectly positive correlation. Correlation is defined as ``the extent
to which to variables are linearly related''. Note that correlation does
not imply causation. Just because two variables are related to each
other does not mean one caused the other. For example, if a study showed
that Ice Cream Sales and Drowning Deaths have a strong positive
correlation, it does not mean that simply because of more ice cream
being sold, more people drowned (that makes no sense!). Instead, there
would be a third variable, known as a confounding variable, which
influences both the dependent and independent variables. In this case,
the confounding variable would be summer heat; with more heat, more ice
cream will be sold and more people will go swimming. Thus, just because
more ice cream was sold does not mean more people would die by drowning.
Correlation does not imply causation.\\
The distribution of correlation is as follows:

\begin{verbatim}
c = -1                  Perfectly Negative
-1 < c <= -0.7          Strong Negative
-0.7 < c <= -0.4        Moderate Negative
-0.4 < c < 0            Weak Negative
c = 0                   No correlation
0 < c < 0.4             Weak Positive
0.4 <= c < 0.7          Moderate Positive
0.7 <= c < 1            Strong Positive
c = 1                   Perfectly Positive
\end{verbatim}

\paragraph{Back to the analysis}\label{back-to-the-analysis}

Here, the correlation is shown to be a little over 0.37, which is
considered a Weak Positive correlation. From this, we can understand
that magnitude does affect the property loss, but only to a limited
degree.\\
But what if we take them apart individually? What can we see? Well,
let's first compare the number summaries of property losses for each
magnitude.

\begin{verbatim}
##          Min.     1st Qu.      Median        Mean     3rd Qu.        Max.
## 1      1000.0     25000.0     60000.0    157519.2    150000.0   5000000.0
## 2      2000.0    218750.0    500000.0   3129930.6   1000000.0 150000000.0
## 3    100000.0    450000.0   1500000.0  12348076.9  20000000.0  50175000.0
## 4   5900000.0  21450000.0  37000000.0  87633333.3 128500000.0 220000000.0
\end{verbatim}

From this table, we can see gradual increase in both the median and mean
property loss with an increasing magnitude. So there does seem to be a
correlation between an increasing magnitude and increasing property
loss. To visualize this, let's put the data into a table:

\begin{verbatim}
##   Magnitude Median $
## 1         1  6.0e+04
## 2         2  5.0e+05
## 3         3  1.5e+06
## 4         4  3.7e+07
\end{verbatim}

and get the correlation between the two columns:

\begin{verbatim}
##            Median $
## Magnitude 0.7946369
\end{verbatim}

We can do the same for mean:

\begin{verbatim}
##   Magnitude     Mean $
## 1         1   157519.2
## 2         2  3129930.6
## 3         3 12348076.9
## 4         4 87633333.3
\end{verbatim}

and get their correlation:

\begin{verbatim}
##              Mean $
## Magnitude 0.8443055
\end{verbatim}

From these, we get vastly different results compared to what we
originally got.\\
A 0.79 correlation for the median and 0.84 correlation for the mean both
represent a Strong Positive Correlation, meaning there is a huge
relationship between the magnitude of the tornado and the damage it has
done. The reason for this discrepancy is because of the data set each of
the methods is looking at. The first method looks at the entire data
set, meaning any lows and highs will cause a lower correlation. However,
the second method looks specifically at the median and mean, which both
combined do a pretty good job (without a large standard deviation from
the mean) at representing the data. Since this method only looks at the
representation of the data rather than the data itself, the correlation
will be higher (and less affected by any outliers).

And now, we move on to the final bivariate relationship.

\subsubsection{Distance Traveled vs.~Property
Loss}\label{distance-traveled-vs.-property-loss}

How does the distance a tornado traveled relate to the property loss
caused by said tornado? Well, there are a lot of factors to take into
consideration before answering that question. Where was this happening:
the urbanized regions or somewhere in the boonies? How fast was the
tornado going {[}with larger speeds and more distance a ton of damage
could be done{]}? So on and so forth.\\
But we don't have any of that information, just the distance. So let's
try to analyze our data with what we've got.

\includegraphics{Tornadoes_files/figure-latex/unnamed-chunk-12-1.pdf}

\subparagraph{Figure B.3: The relationship between the distance traveled
and the property
loss.}\label{figure-b.3-the-relationship-between-the-distance-traveled-and-the-property-loss.}

So what can we see from the scatterplot?\\
Well, visible immediately is the cluster of data points within the 20
mile and \$5.0e+07 box. The data points vary in area after this cluster.
This points to there being a low correlation between the two variables.
Let's do the math, just in case:

\begin{verbatim}
## [1] 0.4469718
\end{verbatim}

So the data shows a 0.45, which means it is a Moderate Positive
Correlation. This shows that, with all points (other than N/A) in
consideration, there is somewhat of a relationship between the distance
a tornado traveled and the damage done.\\
Just for fun, let's exclude the outliers in the data and focus only on
the 20 mile and \$5.0e+07 region:\\
\includegraphics{Tornadoes_files/figure-latex/unnamed-chunk-14-1.pdf}

Now that we have the graph for the region, let's find the correlation
for it:

\begin{verbatim}
## [1] 0.2077407
\end{verbatim}

Wait, what? Why is the correlation lower? Not what you expected,
right?\\
Well, the correlation, .21, is lower, because most of the data is
distributed horizontally, which is linear, but with a very minimal
growth. This shows that their relationship is not actually that strong,
meaning for this range, the distance a tornado traveled does not have a
good relation to the damage it caused.

\begin{itemize}
\tightlist
\item
  Why did the data join the tornado's team?\\
  Because it wanted to be part of a real \(twist\) in the analysis!
\end{itemize}

\end{document}
